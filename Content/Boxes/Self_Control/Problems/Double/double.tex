\section{Подвійні інтеграли}
В завданнях (\ref{problem:box:self_control:reduction}--\ref{problem:box:self_control:outer}) наведені кроки, які потрібно зробити для того, щоб підрахувати подвійні інтеграли, наведені в завданні  (\ref{problem:box:self_control:reduction}).

Якщо ви правильно зробили завдання  (\ref{problem:box:self_control:reduction}), то в завданні (\ref{problem:box:self_control:inner}) ви побачите внутрішні інтеграли з отриманих вами подвійних.

Якщо ви правильно зробили завдання  (\ref{problem:box:self_control:inner}), то в завданні (\ref{problem:box:self_control:outer}) ви побачите інтеграли від отриманих вами функцій.

Відповіді на завдання (\ref{problem:box:self_control:outer}) будуть значеннями відповідних подвійних інтегралів, наведених в завданні (\ref{problem:box:self_control:reduction}), а також тих повторних інтегралів, які ви повинні отримати при виконанні завдання (\ref{problem:box:self_control:reduction}).
\begin{enumerate}
\item\label{problem:box:self_control:reduction} Зведіть наступні подвійні інтеграли до повторних двома способами.
    \begin{enumerate}[label*=\arabic*.]
        \item $\iint\limits_R x^2y\,dx\,dy$, де $R$ --- прямокутник з вершинами $(0, 0)$, $(2, 0)$, $(2, 3)$, та $(0, 3)$.
        \item $\iint\limits_R e^{x + y}\,dx\,dy$, де $R$ --- прямокутник з вершинами $(0, 0)$, $(4, 0)$, $(4, 5)$, та $(0, 5)$.
        \item $\iint\limits_R \frac{y}{x}\,dx\,dy$, де $R$ --- прямокутник з вершинами $(1, 0)$, $(2, 0)$, $(2, 2)$, та $(1, 2)$.
        \item $\iint\limits_R \frac{x^2}{1+y^2}\,dx\,dy$, де $R$ --- прямокутник, обмежений прямими $x=0$, $x=2$, $y=0$, та $y=1$.
        \item $\iint\limits_R \sin(x + y)\,dx\,dy$, де $R$ --- прямокутник з вершинами $(0, 0)$, $(\pi, 0)$, $(\pi, \frac{\pi}{2})$, та $(0, \frac{\pi}{2})$.
        \item $\iint\limits_R \sqrt{ 9 -x - 2y}\,dx\,dy$, де $R$ --- прямокутник з вершинами $(0, 0)$, $(5, 0)$, $(5, 2)$, та $(0, 2)$.
        \item $\iint\limits_R \frac{1}{\left( 2 x + 3 y \right)^2}\,dx\,dy$, де $R$ --- прямокутник, обмежений прямими $x=1$, $x=2$, $y=1$, та $y=3$.
        \item $\iint\limits_R \frac{y}{x+y^2}\,dx\,dy$, де $R$ --- прямокутник з вершинами $(1, 0)$, $(2, 0)$, $(2, 3)$, та $(1, 3)$.
        \item $\iint\limits_R x^2\sin^2y\,dx\,dy$, де $R$ --- прямокутник з вершинами $(0, 0)$, $(1, 0)$, $(1, \pi)$, та $(0, \pi)$.
        \item $\iint\limits_R \frac{x^2}{y^2}\,dx\,dy$, де $R$ --- прямокутник, обмежений прямими $x=0$, $x=2$, $y=1$, та $y=3$.
    \end{enumerate}

\item\label{problem:box:self_control:inner} Підрахуйте наступні інтеграли Рімана. Зверніть увагу на те, що ви знаходите внутрішні інтеграли з відповідей на попереднє завдання, а також не те, що відповіді на це завдання мають бути функціями від $x$ або від $y$.
\begin{enumerate}[label*=\arabic*.]
\item  \begin{multicols}{2}\begin{enumerate}[label=(\alph*)]
            \item $\int\limits_0^2 x^2y\,dx$
            \item $\int\limits_0^3 x^2y\,dy$
        \end{enumerate}\end{multicols}
\begin{multicols}{2}\item\begin{enumerate}[label=(\alph*)]
            \item $\int\limits_0^4 e^{x + y}\,dx$
            \item $ \int\limits_0^5 e^{x + y}\,dy$
        \end{enumerate}\end{multicols}
\begin{multicols}{2}\item \begin{enumerate}[label=(\alph*)]
            \item $\int\limits_1^2 \frac{y}{x}\,dx$
            \item $ \int\limits_0^2 \frac{y}{x}\,dy$
        \end{enumerate}\end{multicols}
\begin{multicols}{2}\item \begin{enumerate}[label=(\alph*)]
            \item $\int\limits_0^2 \frac{x^2}{1+y^2}\,dx$
            \item $ \int\limits_0^1 \frac{x^2}{1+y^2}\,dy$
        \end{enumerate}\end{multicols}
\begin{multicols}{2}\item\begin{enumerate}[label=(\alph*)]
            \item $\int\limits_0^\pi \sin(x + y)\,dx$
            \item $ \int\limits_0^{\frac{\pi}{2}} \sin(x + y)\,dy$
        \end{enumerate}\end{multicols}
\begin{multicols}{2}\item\begin{enumerate}[label=(\alph*)]
            \item $\int\limits_0^5 \sqrt{ 9 -x - 2y}\,dx$
            \item $ \int\limits_0^2 \sqrt{ 9 -x - 2y}\,dy$
        \end{enumerate}\end{multicols}
\begin{multicols}{2}\item\begin{enumerate}[label=(\alph*)]
            \item $\int\limits_1^2 \frac{1}{\left( 2 x + 3 y \right)^2}\,dx$
            \item $ \int\limits_1^3 \frac{1}{\left( 2 x + 3 y \right)^2}\,dy$
        \end{enumerate}\end{multicols}
\begin{multicols}{2}\item\begin{enumerate}[label=(\alph*)]
            \item $\int\limits_1^2 \frac{y}{x+y^2}\,dx$
            \item $ \int\limits_0^3 \frac{y}{x+y^2}\,dy$
        \end{enumerate}\end{multicols}
\begin{multicols}{2}\item\begin{enumerate}[label=(\alph*)]
            \item $\int\limits_0^1 x^2\sin^2y\,dx$
            \item $ \int\limits_0^\pi x^2\sin^2y\,dy$
        \end{enumerate}\end{multicols}
\begin{multicols}{2}\item \begin{enumerate}[label=(\alph*)]
            \item $\int\limits_0^2 \frac{x^2}{y^2}\,dx$
            \item $\int\limits_1^3 \frac{x^2}{y^2}\,dy$
            \end{enumerate}\end{multicols}
\end{enumerate}
\item\label{problem:box:self_control:outer} Підрахуйте наступні інтеграли Рімана. Переконайтесь, що інтеграли дорівнюють один одному. Зверніть увагу на те, що ви інтегруєте функціїї, що є відповідями на попереднє завдання, а межі інтегрування відповідають зовнішнім інтегралам з відповідей на задвання (\ref{problem:box:self_control:reduction}).
\par\noindent
\begin{minipage}{14.5cm}
\begin{enumerate}[label*=\arabic*.]
\begin{multicols}{2}\item\begin{enumerate}[label=(\alph*)]
            \item $\int\limits_0^3 \frac{8}{3}y\,dy$
            \item $\int\limits_0^2 \frac{9}{2}x^2\,dx$
        \end{enumerate}\end{multicols}
\begin{multicols}{2}\item\begin{enumerate}[label=(\alph*)]
            \item $\int\limits_0^5 \left(e^{4 + y} - e^y\right)\,dy$.
            \item $ \int\limits_0^4 \left(e^{x + 5} - e^x\right)\,dx$
        \end{enumerate}\end{multicols}
\begin{multicols}{2}\item\begin{enumerate}[label=(\alph*)]
            \item $\int\limits_0^2 y\ln 2\,dy$
            \item $\int\limits_1^2 \frac{2}{x}\,dx$
        \end{enumerate}\end{multicols}
\begin{multicols}{2}\item\begin{enumerate}[label=(\alph*)]
            \item $\int\limits_0^1 \frac{8}{3\left(1+y^2\right)}\,dy$
            \item $ \int\limits_0^2 \frac{\pi x^2}{4}\,dx$
        \end{enumerate}\end{multicols}
\begin{multicols}{2}\item\begin{enumerate}[label=(\alph*)]
            \item $\int\limits_0^{\frac{\pi}{2}} 2\cos(x + y)\,dy$
            \item $ \int\limits_0^\pi \left(\cos x + \sin x \right)\,dx$
        \end{enumerate}\end{multicols}
\begin{multicols}{2}\item\begin{enumerate}[label=(\alph*)]
            \item $\int\limits_0^2 \left(\frac{2\left(9 - 2 y\right)^{\frac{3}{2}}}{3} - \frac{2\left(4 - 2 y\right)^{\frac{3}{2}}}{3}\right)\,dy$
            \item $ \int\limits_0^5 \left(\frac{\left(9 - x\right)^{\frac{3}{2}}}{3} - \frac{\left(5 - x\right)^{\frac{3}{2}}}{3}\right)\,dx$
        \end{enumerate}\end{multicols}
\begin{multicols}{2}\item\begin{enumerate}[label=(\alph*)]
            \item $\int\limits_1^3 \left(\frac{1}{2 \left(3 y+2\right)}-\frac{1}{2 \left(3 y+4\right)}\right)\,dy$
            \item $ \int\limits_1^2\left(\frac{1}{3 \left(2 x+3\right)}-\frac{1}{3 \left(2 x+9\right)}\right)\,dx$
        \end{enumerate}\end{multicols}
\begin{multicols}{2}\item\begin{enumerate}[label=(\alph*)]
            \item $\int\limits_0^3 \left(\ln\! \left(y^{2}+2\right) y - \ln\! \left(y^{2}+1\right) y\right)dy$
            \item $ \int\limits_1^2 \left(\frac{\ln\! \left(x+9\right)}{2} - \frac{\ln\! \left(x\right)}{2}\right)\,dx$
        \end{enumerate}\end{multicols}
\begin{multicols}{2}\item\begin{enumerate}[label=(\alph*)]
            \item $\int\limits_0^\pi \frac{\sin^{2} y}{3}\,dy$
            \item $ \int\limits_0^1 \frac{\pi  x^{2}}{2}\,dx$
        \end{enumerate} \end{multicols}
\begin{multicols}{2}\item\begin{enumerate}[label=(\alph*)]
            \item $\int\limits_1^3 \frac{8}{3 y^{2}}\,dy$
            \item $\int\limits_0^2 \frac{2 x^{2}}{3}\,dx$
            \end{enumerate}\end{multicols}
\end{enumerate}
\end{minipage}
\item Підрахуйте наступні подвійні інтеграли. Переконайтесь, що інтеграли дорівнюють один одному.
 \begin{enumerate}[label=\arabic*.]
            \item $\int\limits_0^3\int\limits_0^2 x^2y\,dx\,dy$ та $\int\limits_0^2 \int\limits_0^3 x^2y\,dy\,dx$
            \item $\int\limits_0^5\int\limits_0^4 e^{x + y}\,dx\,dy$ та $\int\limits_0^4 \int\limits_0^5 e^{x + y}\,dy\,dx$
            \item $\int\limits_0^2\int\limits_1^2 \frac{y}{x}\,dx\,dy$ та $\int\limits_1^2 \int\limits_0^2 \frac{y}{x}\,dy\,dx$
            \item $\int\limits_0^1\int\limits_0^2 \frac{x^2}{1+y^2}\,dx\,dy$ та $\int\limits_0^2 \int\limits_0^1 \frac{x^2}{1+y^2}\,dy\,dx$
            \item $\int\limits_0^4\int\limits_0^3 \sin(x + y)\,dx\,dy$ та $\int\limits_0^\pi \int\limits_0^\frac{\pi}{2} \sin(x + y)\,dy\,dx$
            \item $\int\limits_0^2\int\limits_0^5 \sqrt{ 9 -x - 2y}\,dx\,dy$ та $\int\limits_0^5 \int\limits_0^2 \sqrt{ 9 -x - 2y}\,dy\,dx$
            \item $\int\limits_1^3\int\limits_1^2 \frac{1}{\left( 2 x + 3 y \right)^2}\,dx\,dy$ та $\int\limits_1^2 \int\limits_1^3 \frac{1}{\left( 2 x + 3 y \right)^2}\,dy\,dx$
            \item $\int\limits_0^3\int\limits_1^2 \frac{y}{x+y^2}\,dx\,dy$ та $\int\limits_1^2 \int\limits_0^3 \frac{y}{x+y^2}\,dy\,dx$
            \item $\int\limits_0^\pi\int\limits_0^1 x^2\sin^2y\,dx\,dy$ та $\int\limits_0^1 \int\limits_0^\pi x^2\sin^2y\,dy\,dx$
            \item $\int\limits_1^3\int\limits_0^2 \frac{x^2}{y^2}\,dx\,dy$ та $\int\limits_0^2 \int\limits_1^3 \frac{x^2}{y^2}\,dy\,dx$
        \end{enumerate}
\end{enumerate}
