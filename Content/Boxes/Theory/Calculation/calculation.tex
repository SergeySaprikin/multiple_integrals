\section{Обчислення кратних інтегралів по гіперпрямокутниках}
Обчислення кратних інтегралів відбувається шляхом зведення них до повторних інтегралів Рімана на підставі наступних теорем.
\begin{theorem}
Нехай задані брус
\[
Q = \segment{a_1}{b_1}\times\segment{a_2}{b_2}\times\ldots\times\segment{a_m}{b_m} \subset \eucl{m}
\]
і функція $f\colon Q\to \R$, що неперервна на цьому брусі. Тоді для довільного $k$ між $1$ і $m$ виконуються наступні твердження:
\begin{enumerate}[(1)]
\item для довільного фіксованого ${\overline{x_k} \in \segment{a_k}{b_k}}$ функція
\[
f\left(x_1, \ldots, x_{k-1}, \overline{x_k}, x_{k+1}, \ldots, x_m \right)
\]
неперервна на брусі
\[
Q_k = \segment{a_1}{b_1}\times\ldots\times\segment{a_{k-1}}{b_{k-1}}\times\segment{a_{k+1}}{b_{k+1}}\times\ldots\times\segment{a_m}{b_m} \subset \eucl{m-1};
\]
\item має місце співвідношення
\[
\begin{array}{lr}
\int\limits_Q f(x)d x = \\
=\int\limits_{a_k}^{b_k}\left(\int\limits_{Q_k}f\left(x_1, \ldots, x_{k-1}, x_k, x_{k+1}, \ldots, x_m \right)d x_1 \ldots d x_{k-1} d x_{k+1} \ldots d x_m\right)d x_k.
\end{array}
\]
\end{enumerate}
\end{theorem}
Застосовуючи послідовно цю теорему для ${k = m, m - 1, \ldots, 2, 1}$, отримаємо наступний
\begin{corollary}
В умовах останньої теореми має місце формула
\[
\begin{array}{lr}
\int\limits_Q f(x)d x = \\
=\int\limits_{a_{m}}^{b_{m}}\left(
\int\limits_{a_{m-1}}^{b_{m-1}}\left(
\ldots\left(
\int\limits_{a_1}^{b_1}
f\left(x_1, x_2, \ldots, x_m \right)
d x_1 \right)
\ldots\right)
d x_{m-1} \right)
d x_m .
\end{array}
\]
\end{corollary}
