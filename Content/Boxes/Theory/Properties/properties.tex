\section{Властивості кратних інтегралів по гіперпрямокутниках}
\begin{enumerate}
\item Інтеграл від константи.
\begin{intextProposition}
Для довільної дійсної константи $c$ стала функція ${f(x) \equiv c}$ інтегровна на довільному брусі $Q$, причому
\[
\int\limits_{Q} c d x = c\m\left(Q\right).
\]
\end{intextProposition}
\item Лінійність
\begin{intextProposition}
Якщо обидві функції ${f\colon Q \to \R}$ і ${g\colon Q \to \R}$ неперервні на брусі $Q$, то для довільних дійсних чисел $\alpha$ і $\beta$ виконується рівність
\[
\int\limits_{Q} \left(\alpha f(x) + \beta g(x)\right)d x = \alpha\int\limits_{Q} f(x) d x + \beta\int\limits_{Q} g(x) d x.
\]
\end{intextProposition}
\item Аддитивність
\begin{intextProposition}
Якщо брус $Q$ є об'єднанням двох брусів --- ${Q = Q_1 \cup Q_2}$, причому бруси $Q_1$ і $Q_2$ не мають спільних внутрішніх точок, а функція ${f\colon Q \to \R}$ неперервна на $Q$, то
\[
\int\limits_{Q} f(x) d x = \int\limits_{Q_1} f(x) d x + \int\limits_{Q_2} f(x) d x.
\]
\end{intextProposition}
\begin{remark}
Окільки функція ${f}$ неперервна на $Q$, то вона неперервна на $Q_1$ і $Q_2$, а значить, всі три інтеграли існують.
\end{remark}
\item Невід'ємність
\begin{intextProposition}
Якщо функція ${f\colon Q \to \R}$ неперервна на брусі $Q$ і ${\forall x\in Q\ f(x)\geq 0}$, то ${\int\limits_{Q} f(x) d x \geq 0.}$
\end{intextProposition}
\item Монотонність
\begin{intextProposition}
Якщо обидві функції ${f\colon Q \to \R}$ і ${g\colon Q \to \R}$ неперервні на брусі $Q$ і ${\forall x\in Q\ f(x)\geq g(x)}$, то ${\int\limits_{Q} f(x) d x \geq \int\limits_{Q} g(x) d x.}$
\end{intextProposition}
\item Модуль інтеграла
\begin{intextProposition}
Якщо функція ${f\colon Q \to \R}$ неперервна на брусі $Q$, то
\[
\left|\int\limits_{Q} f(x) d x\right| \leq \int\limits_{Q} \left|f(x)\right| d x.
\]
\end{intextProposition}
\item Теорема про середнє значення
\begin{intextProposition}
Якщо функція ${f\colon Q \to \R}$ неперервна на брусі $Q$, то
\[
\exists \theta\in Q\colon \int\limits_{Q} f(x) d x = f\left(\theta\right)\m\left(Q\right).
\]
\end{intextProposition}
\end{enumerate}
