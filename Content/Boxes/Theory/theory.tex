\chapter{Теоретичні відомості}
В цьому розділі ми коротко розглянемо, як визначаються кратні інтеграли по гіперпрямокутниках, які вони мають властивості і на підставі яких теорем відбувається їх обчислення. Більш детально з відповідним матеріалом можна ознайомитись в наступних джерелах:
\begin{itemize}
\item \cite{Dor94v2}, глава 14, \$1;
\end{itemize}

\subimport{Definitions}{definitions}
\subimport{Properties}{properties}
\subimport{Calculation}{calculation}
