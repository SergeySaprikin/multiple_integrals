\section{Вимірні множини}\label{part:measure}

Наша наступна мета --- визначити кратні інтеграли не тільки по гіперпрямокутниках, а і по множинах більш широкого класу. Цей клас --- клас вимірних за Жорданом множин. До визначення цього класу ми зараз і переходимо.
\subsection{Розбиття простору \eucl{m}}
\begin{definition}
Розбиттям нульового порядку простору \eucl{m} будемо називати зображення простору \eucl{m} у вигляді об'єднання всіх можливих брусів. сторони яких є відрізками довижини 1 з цілими координатами:
\[
\eucl{m} = \bigcup\limits_{n_1, n_2, \ldots, n_m\in\Z}\segment{n_1}{n_1+1}\times\segment{n_2}{n_2+1}\times\ldots\times\segment{n_m}{n_m+1}.
\]
\end{definition}
\begin{definition}
Розбиттям $k$--го порядку для ${k\in\N}$ простору \eucl{m} будемо називати зображення простору \eucl{m} у вигляді об'єднання всіх можливих брусів, сторони яких є відрізками довижини ${\frac{1}{2^k}}$ з координатами, які можуть бути надані у вигляді раціонального дробу (можливо, скоротного) ${\frac{n}{2^k}}$, ${n\in\Z}$:
\[
\eucl{m} = \bigcup\limits_{n_1, n_2, \ldots, n_m\in\Z}\segment{\frac{n_1}{2^k}}{\frac{n_1+1}{2^k}}\times\segment{\frac{n_2}{2^k}}{\frac{n_2+1}{2^k}}\times\ldots\times\segment{\frac{n_m}{2^k}}{\frac{n_m+1}{2^k}}.
\]
\end{definition}
Множину всіх брусів розбиття порядку $k$, ${k\in\N\cup\{0\}}$ будемо позначати ${\pi^{(k)}}$:
\[
\pi^{(k)} =
\left\{
\left.
\segment{\frac{n_1}{2^k}}{\frac{n_1+1}{2^k}}\times\segment{\frac{n_2}{2^k}}{\frac{n_2+1}{2^k}}\times\ldots\times\segment{\frac{n_m}{2^k}}{\frac{n_m+1}{2^k}}
\right| n_1, n_2, \ldots, n_m\in\Z
\right\}.
\]
\subsection{Міра Жордана}
\hyperref[def:box]{Нагадаємо}, що мірою брусу
\[
Q = \segment{a_1}{b_1}\times\segment{a_2}{b_2}\times\ldots\times\segment{a_m}{b_m} \subset \eucl{m},
\]
називають число
\[
\m(Q) = \left(b_1-a_1\right)\left(b_2 - a_2\right)\ldots\left(b_m-a_m\right) = \prod\limits_{j=1}^{m}\left(b_j - a_j\right).
\]
\begin{definition}
Якщо множина ${A\subset\eucl{m}}$ є об'єднанням скінченої кількості брусів, внутрішності яких попарно не перетинаються,
\[
A = \bigcup\limits_{j=1}^nQ_j,
\]
то мірою $A$ будемо називати суму мір цих брусів:
\[
\m\left(A\right) = \sum\limits_{j=1}^n\m\left(Q_j\right).
\]
Зокрема, мірою порожньої множини будемо вважати 0:
\[
\m(\emptyset) = 0.
\]
\end{definition}
Це означення, зокрема, можна застосовувати для тих випадків, коли ${Q_1}$, ${Q_2}$, ${\ldots}$, ${Q_n}$ --- різні бруси певного розбиття ${\pi^{(k)}}$.

Нехай тепер ${A\subset\eucl{m}}$ --- довільна обмежена множина, ${\pi^{(k)}}$ --- розбиття \eucl{m}, $k\in\N\cup\{0\}$. Введемо наступні позначення:
\begin{itemize}\label{partition_sets}
\item $A_{(k)}$ --- об'єднання всіх брусів з ${\pi^{(k)}}$, які є підмножиною $A$:
\[
A_{(k)} = \bigcup\limits_{\tiny{\begin{array}{c}Q\in\pi^{(k)}\\Q\subset A\end{array}}}Q;
\]
\item $A^{(k)}$ --- об'єднання всіх брусів з ${\pi^{(k)}}$, які мають непорожній перетин з $A$:
\[
A_{(k)} = \bigcup\limits_{\tiny{\begin{array}{c}Q\in\pi^{(k)}\\Q\cap A\neq \emptyset\end{array}}}Q;
\]
\item $\Delta A^{(k)}$ --- різниця множин $A^{(k)}$ і $A_{(k)}$:
\[
\Delta A_{(k)} = A^{(k)} \setminus A_{(k)}.
\]
\end{itemize}
У випадках, коли описаних вище брусів не існує, будемо вважати відповідну множину порожньою.
Множини ${A_{(k)}}$, ${A^{(k)}}$ і ${\Delta A_{(k)}}$ мають наступні властивості:
\begin{itemize}
\item ${A_{(k)} \subset A \subset A^{(k)}}$,
\item ${A_{(k)} \subset A_{(k+1)}}$,
\item ${A^{(k+1)} \subset A^{(k)}}$.
\end{itemize}
Оскільки всі три множини ${A_{(k)}}$, ${A^{(k)}}$ і ${\Delta A_{(k)}}$ є об'єднаннями брусів розбиття ${\pi^{(k)}}$, то визначені їх міри ${\m\left(A_{(k)}\right)}$, ${\m\left(A^{(k)}\right)}$ і ${\m\left(\Delta A_{(k)}\right)}$, причому зозначення і властивостей множин випливають властивості їх мір:
\begin{itemize}
\item ${0 \leq \m\left(A_{(k)}\right) \leq \m\left(A^{(k)}\right)}$,
\item ${\m\left(A_{(k)}\right) \leq \m\left(A_{(k+1)}\right)}$,
\item ${\m\left(A^{(k+1)}\right) \leq \m\left(A^{(k)}\right)}$,
\item ${\m\left(\Delta A_{(k)}\right) = \m\left(A^{(k)}\right) - \m\left(A_{(k)}\right)}$.
\end{itemize}
З цих властивостей, зокрема, випливає, що обидві послідовності ${\left\{\m\left(A_{(k)}\right)\right\}}$ і ${\left\{\m\left(A^{(k)}\right)\right\}}$ є обмеженими і монотонними, а значить, збіжними.
\begin{definition}
Внутрішньою мірою обмеженої множини ${A\subset\eucl{m}}$ будемо називати число
\[
\m_{*}\left(A\right) = \lim\limits_{k\to\infty}\m\left(A_{(k)}\right).
\]
Зовнішньою мірою множини ${A}$ будемо називати число
\[
\m^{*}\left(A\right) = \lim\limits_{k\to\infty}\m\left(A^{(k)}\right).
\]
\end{definition}
З властивостей міри множин ${A_{(k)}}$ і ${A^{(k)}}$ випливає, що
\[
0 \leq \m_{*}\left(A\right) \leq \m^{*}\left(A\right)
\]
для довільної обмеженої множини $A$.
\begin{definition}
Обмежену множину ${A\subset\eucl{m}}$ будемо називати вимірною за Жорданом, якщо
\[
\m_{*}\left(a\right) = \m^{*}\left(a\right).
\]
В цьому випадку спільне значення ${\m_{*}\left(a\right) = \m^{*}\left(a\right)}$ будемо називати ($m$--вимірною) мірою Жордана множини $A$ і будемо позначати ${\m\left(A\right)}$:
\[
\m\left(A\right) =\m_{*}\left(A\right) = \m^{*}\left(A\right).
\]
\end{definition}
Вимірні за Жорданом множини мають наступні властивості:
\begin{itemize}
\item\label{prop:measrable_sets:1} якщо $A$ і $B$ --- вимірні за Жорданом, то і ${A\cup B}$ також є вимірною за Жорданом;
\item якщо $A$ і $B$ --- вимірні за Жорданом, то і ${A\cap B}$ також є вимірною за Жорданом;
\item якщо $A$ і $B$ --- вимірні за Жорданом, то і ${A\setminus B}$ також є вимірною за Жорданом.
\end{itemize}
Міра Жордана має наступні властивості:
\begin{enumerate}
\item ${\m\left(A\right)\geq 0}$ для довільної вимрної за Жорданом множини ${A}$;
\item {\bf напівадидивність}:
${\m\left(A\cup B\right) \leq \m\left(A\right) + \m\left(B\right)}$ для довільних вимрних за Жорданом множин ${A}$ і ${B}$;
\item {\bf адидивність}:
${\m\left(A\cup B\right) = \m\left(A\right) + \m\left(B\right)}$ для довільних вимрних за Жорданом множин ${A}$ і ${B}$, внутрішності яких не перетинаються;
\item {\bf монотонність}: ${\m\left(A\right)\leq \m\left(B\right)}$ для довільних вимрних за Жорданом множин ${A\subset B}$.
\end{enumerate}
Важливим класом вимвірних множин є циліндричні множини, визначені в наступному означенні.
\begin{definition}
Нехай ${m\geq2}$. У просторі \eucl{m-1} розглянемо множину ${A \subset \eucl{m-1}}$ і дві функції
\[
u_1,u_2\colon A\to \R,
\]
які задовольняють наступну властивість:
\[
\forall \left(x_1, x_2, \ldots, x_{m-1}\right)\in A\ u_1\left(x_1, x_2, \ldots, x_{m-1}\right)\leq u_2\left(x_1, x_2, \ldots, x_{m-1}\right).
\]
Тоді множину
\[
C =
\left\{
\left(x_1, x_2, \ldots, x_{m-1}, x_m\right)
\left|
\begin{array}{c}
\left(x_1, x_2, \ldots, x_{m-1}\right)\in A,\\
u_1\left(x_1, x_2, \ldots, x_{m-1}\right)\leq x_m \leq u_2\left(x_1, x_2, \ldots, x_{m-1}\right)
\end{array}
\right.
\right\}
\]
будемо називати циліндричною в напрямку осі ${Ox_m}$ з основою $A$. Основу циліндричної множини $C$ будемо позначати ${\ba C}$:
\[
A = \ba C.
\]
\end{definition}
Вимірність широкого класу циліндричних множин встановлюється наступною теоремою.
\begin{theorem}
Нехай $C$ --- циліндрична множина, що визначається основою $\ba C$ і функціями ${u_1, u_2\colon \ba C\to \R.}$ Якщо
\begin{enumerate}
\item $\ba C$ --- компактна вимірна підмножина \eucl{m-1},
\item функції $u_1$ і $u_2$ неперервні на $\ba C$,
\end{enumerate}
то множина $C$ --- компактна і вимірна.
\end{theorem}
