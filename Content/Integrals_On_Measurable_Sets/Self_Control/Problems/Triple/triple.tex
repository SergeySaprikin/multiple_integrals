\section{Потрійні інтеграли}
В завданнях (\ref{problem:measurable_sets:self_control:triple:reduction}--\ref{problem:measurable_sets:self_control:triple:outer}) описані кроки, які потрібно зробити для того, щоб підрахувати потрійні інтеграли, наведені в завданні  (\ref{problem:measurable_sets:self_control:triple:reduction}).

Якщо ви правильно зробили завдання  (\ref{problem:measurable_sets:self_control:triple:reduction}), то в завданні (\ref{problem:measurable_sets:self_control:triple:inner}) ви побачите внутрішні інтеграли з отриманих вами потрійних.

Якщо ви правильно зробили завдання  (\ref{problem:measurable_sets:self_control:triple:inner}), то в завданні (\ref{problem:measurable_sets:self_control:triple:middle}) ви побачите інтеграли від отриманих вами функцій, а межі інтегрування відповідатимуть межам у відповіному інтегралі з отриманним вами в завданні (\ref{problem:measurable_sets:self_control:triple:reduction} повторних інтегралів.

Так само, якщо ви правильно зробили завдання  (\ref{problem:measurable_sets:self_control:triple:middle}), то в завданні (\ref{problem:measurable_sets:self_control:triple:outer}) ви побачите інтеграли від отриманих вами функцій, а межі інтегрування відповідатимуть межам у відповіному інтегралі з отриманним вами в завданні (\ref{problem:measurable_sets:self_control:triple:reduction} повторних інтегралів.

Відповіді на завдання (\ref{problem:measurable_sets:self_control:triple:outer}) будуть значеннями відповідних потрійних інтегралів, наведених в завданні (\ref{problem:measurable_sets:self_control:triple:reduction}), а також тих повторних інтегралів, які ви повинні отримати при виконанні завдання (\ref{problem:measurable_sets:self_control:triple:reduction}).
\begin{enumerate}
\item\label{problem:measurable_sets:self_control:triple:reduction} Зведіть наступні потрійні інтеграли до повторних таким чином, щоб в повторному інтегралі інтегрування відбувалось спочатку по $y$, потім по $x$, а в кінці по $z$.
    \begin{enumerate}[label*=\arabic*.]
        \item $\iiint\limits_D z \,dx\,dy\,dz$, де $D$ --- прямокутний паралелепіпед, обмежений площинами $x = 0$, $y = 0$, $z = 0$, $x = 1$, $y = 2$, $z = 3$.
        \item  $\iiint\limits_{V} \left(x^2 + y^2 + z^2\right) \, dx \, dy \, dz$, де $V$ --- прямокутний паралелепіпед, обмежений площинами $x = 0$, $y = 0$, $z = 0$, $x = 2$, $y = 3$, $z = 4$.
        \item $\iiint\limits_{W} 8 z^{2} x {\mathrm e}^{2 x y z} \, dx \, dy \, dz$, де $W$ --- прямокутний паралелепіпед з вершинами $(0,0,0)$, $(1,0,0)$, $(0,2,0)$, $(0,0,3)$.
        \item $\iiint\limits_{D} \left(x + 2y + 3z\right) \, dx \, dy \, dz$, де $D$ --- прямокутний паралелепіпед, заданий нерівностями $0 \leq x \leq 2$, $1 \leq y \leq 4$, $2 \leq z \leq 5$.
        \item  $\iiint\limits_{P} x z^{2} \cos\left(x y z\right) \, dx \, dy \, dz$, де $P$ --- прямокутний паралелепіпед, визначений координатами $(0,0,0)$, $(1,0,0)$, $(0,\pi,0)$, $(0,0,3)$.
        \item $\iiint\limits_{R} \left(3x + y^2 + 2z\right) \, dx \, dy \, dz$, де $R$ --- прямокутний паралелепіпед, у якому $0 \leq x \leq 1$, $0 \leq y \leq 2$, $0 \leq z \leq 3$.
        \item $\iiint\limits_{S} \left(xy + 2yz + 3zx\right) \, dx \, dy \, dz$, де $S$ --- прямокутний паралелепіпед, обмежений площинами $x = 0$, $y = 0$, $z = 0$, $x = 3$, $y = 2$, $z = 1$.
        \item $\iiint\limits_{T} \frac{1}{xyz} \, dx \, dy \, dz$, де $T$ --- прямокутний паралелепіпед, визначений нерівностями $1 \leq x \leq 2$, $2 \leq y \leq 4$, $3 \leq z \leq 5$.
        \item $\iiint\limits_{U} \left(x^3 + y^3 + z^3\right) \, dx \, dy \, dz$, де $U$ --- прямокутний паралелепіпед, з вершинами $(0,0,0)$, $(1,0,0)$, $(0,1,0)$, $(0,0,1)$.
        \item  $\iiint\limits_{V} \left(2xy + z^2\right) \, dx \, dy \, dz$, де $V$ --- прямокутний паралелепіпед, описаний площинами $x = 0$, $y = 0$, $z = 0$, $x = 2$, $y = 1$, $z = 3$.
    \end{enumerate}
\item \label{problem:measurable_sets:self_control:triple:inner}Підрахуйте наступні інтеграли Рімана. Зверніть увагу на те, що ви знаходите внутрішні інтеграли з відповідей на попереднє завдання, а також не те, що відповіді на це завдання можуть залежати від $x$ та від $z$.
\begin{multicols}{2}
    \begin{enumerate}[label*=\arabic*.]
        \item $\int\limits_0^2 z \,dy$.
        \item $\int\limits_0^3 \left(x^2 + y^2 + z^2\right) \,dy$.
        \item $\int\limits_0^2 8 z^{2} x {\mathrm e}^{{2}{}{x}{}{y}{}{z}} \,dy$.
        \item $\int\limits_1^4 \left(x + 2y + 3z\right) \,dy$.
        \item $\int\limits_0^\pi {x}{}{z^{{2}}}{}{\cos}{\left({x}{}{y}{}{z}\right)} \,dy$.
        \item $\int\limits_0^2 \left(3x + y^2 + 2z\right) \,dy$.
        \item $\int\limits_0^2 \left(xy + 2yz + 3zx\right) \,dy$.
        \item $\int\limits_2^4 \frac{1}{xyz} \,dy$.
        \item $\int\limits_0^1 \left(x^3 + y^3 + z^3\right) \,dy$.
        \item $\int\limits_0^1 \left(2xy + z^2\right) \,dy$.
    \end{enumerate}
\end{multicols}
\item\label{problem:measurable_sets:self_control:triple:middle} Підрахуйте наступні інтеграли Рімана. Зверніть увагу на те, що ви інтегруєте функціїї, що є відповідями на попереднє завдання, а межі інтегрування збігаються з межами відповідних повторних інтегралів з відповідей на задвання (\ref{problem:measurable_sets:self_control:triple:reduction}). Отримані вами функції можуть залежати від $x$.
\begin{multicols}{2}
    \begin{enumerate}[label*=\arabic*.]
        \item $\int\limits_0^1 2z \,dx$.
        \item $\int\limits_0^2 \left(3 x^{2}+3 z^{2}+9\right) \,dx$.
        \item $\int\limits_0^1 \left(-4 z+4 z {\mathrm e}^{4 x z}\right) \,dx$.
        \item $\int\limits_0^2 \left(3 x+15+9 z\right) \,dx$.
        \item $\int\limits_0^1 {\sin}{\left(x z \pi \right)} {z} \,dx$.
        \item $\int\limits_0^1 \left(\frac{8}{3}+6 x+4 z\right) \,dx$.
        \item $\int\limits_0^3 \left(6 z x+2 x+4 z\right) \,dx$.
        \item $\int\limits_1^2 \frac{\ln\! \left(2\right)}{x z} \,dx$.
        \item $\int\limits_0^1 \left({x^{3}+\frac{1}{4}+z^{3}}\right) \,dx$.
        \item $\int\limits_0^2 \left(z^{2}+x\right) \,dx$.
    \end{enumerate}
\end{multicols}
\item\label{problem:measurable_sets:self_control:triple:outer} Підрахуйте наступні інтеграли Рімана. Зверніть увагу на те, що ви інтегруєте функціїї, що є відповідями на попереднє завдання, а межі інтегрування збігаються з межами відповідних повторних інтегралів з відповідей на задвання (\ref{problem:measurable_sets:self_control:triple:reduction}).
\begin{multicols}{2}
    \begin{enumerate}[label*=\arabic*.]
        \item $\int\limits_0^3 2z \,dz$.
        \item $\int\limits_0^4 \left({6} {z^{2}}{+}{26}\right) \,dz$.
        \item $\int\limits_0^3 \left(-1+{\mathrm e}^{4 z}-4z\right) \,dz$.
        \item $\int\limits_2^5 \left(36 + 18 z\right) \,dz$.
        \item $\int\limits_0^3 \dfrac{1 - \cos\! \left(\pi  z\right)}{\pi} \,dz$.
        \item $\int\limits_0^3\left(\frac{17}{3}+4 z\right) \,dz$.
        \item $\int\limits_0^1 \left({39} {z}{+}{9}\right) \,dz$.
        \item $\int\limits_3^5 \frac{\ln\! \left(2\right)^{2}}{z} \,dz$.
        \item $\int\limits_0^1 \left(\frac{1}{2}+z^{3}\right) \,dz$.
        \item $\int\limits_0^3 \left({2} {z^{2}}{+}{2}\right) \,dz$.
    \end{enumerate}
\end{multicols}
\end{enumerate}
