\section{Означення}
Нагадаємо, що кратні інтеграли по брусах були визначені в розділі~\ref{part:boxes}. Тепер в два кроки ми поширемо клас множин, по яких визначаються кратні інтеграли, від брусів до вимірних множин.
\begin{definition}
Нехай множина ${A\subset\eucl{m}}$ є об'єднанням скінченої кількості брусів
\[
A = \bigcup\limits_{j=1}^nQ_j,
\]
внутрішності яких попарно не перетинаються  (бруси ${Q_j}$ і ${Q_k }$ не мають спільних внутрішнії точок при ${j\neq k}$), а функція ${f\colon A\to \R}$ неперервна на $A$. Інтеграл від функції $f$ по множині $A$ визначимо наступною формулою
\[
\int\limits_Af(x) d x = \sum\limits_{j=1}^n\int\limits_{Q_j}f(x) d x.
\]
\end{definition}
\begin{remark}
Зауважимо, що останнє означення може бути застосоване для множин ${A_{(k)}}$, де $A$ --- довільна обмежена множина (дивись \hyperref[partition_sets]{позначення} в розділі~\ref{part:measure}).
\end{remark}
Виявляється, що має місце наступна лема.
\begin{lemma}
Нехай ${A\subset\eucl{m}}$ --- вимірна множина, а функція ${f\colon A\to\R}$ --- неперервна і обмежена на ${A}$. Тоді послідовність ${\left\{\int\limits_{A_{(k)}}f(x) d x\right\}_{k=1}^\infty}$ є фундаментальною (а значить, збіжною!).
\end{lemma}
Ця лема дає можливість визначити кратні інтеграли по вимірних множинах.
\begin{definition}
Нехай ${A\subset\eucl{m}}$ --- вимірна множина, а функція ${f\colon A\to\R}$ --- неперервна і обмежена на ${A}$. Число
\[
\int\limits_{A}f(x) d x = \lim\limits_{k\to\infty}\int\limits_{A_{(k)}}f(x) d x
\]
будемо називати ${m}$--кратним інтегралом від функції  по множині ${A}$. Якщо ${\m\left(A\right) = 0}$, то ${A_{(k)} = \emptyset}$ для всіх ${k\in\N}$, тому в цьому випадку природньо поклдасти за означенням ${\int\limits_{A}f(x) d x = 0}$.
\end{definition}

Властивості кратних інтегралів по вимірних множинах аналогічні властивостям кроатних інтегралів по гіперпрямокутникам.
\begin{enumerate}
\item Інтеграл від константи.
\begin{intextProposition}
Для довільної дійсної константи $c$ стала функція ${f(x) \equiv c}$ інтегровна на довільній вимірниій множині $A$, причому
\[
\int\limits_{A} c d x = c\m\left(Q\right).
\]
\end{intextProposition}
\item Лінійність
\begin{intextProposition}
Якщо обидві функції ${f\colon A \to \R}$ і ${g\colon Q \to \R}$ неперервні і обмежені на вимірній множині $A$, то для довільних дійсних чисел $\alpha$ і $\beta$ виконується рівність
\[
\int\limits_{A} \left(\alpha f(x) + \beta g(x)\right)d x = \alpha\int\limits_{A} f(x) d x + \beta\int\limits_{A} g(x) d x.
\]
\end{intextProposition}
\item Аддитивність
\begin{intextProposition}
Якщо множина $A$ є об'єднанням двох вимірних множин --- ${A = A_1 \cup A_2}$, причому вимірні множини $A_1$ і $A_2$ не мають спільних внутрішніх точок, а функція ${f\colon A \to \R}$ неперервна і обмежена на $A$, то
\[
\int\limits_{A} f(x) d x = \int\limits_{A_1} f(x) d x + \int\limits_{A_2} f(x) d x.
\]
\end{intextProposition}
\begin{remark}
Окільки $A_1$ і $A_2$ --- вимірні множини, то і ${A = A_1 \cup A_2}$ --- вимірна множина за \hyperref[prop:measrable_sets:1]{властивістю вимірних множин}.
\end{remark}
\item Невід'ємність
\begin{intextProposition}
Якщо функція ${f\colon A \to \R}$ неперервна і обмежена на вимірній множині $A$ і ${\forall x\in A\ f(x)\geq 0}$, то ${\int\limits_{A} f(x) d x \geq 0.}$
\end{intextProposition}
\item Монотонність
\begin{intextProposition}
Якщо обидві функції ${f\colon A \to \R}$ і ${g\colon A \to \R}$ неперервні і обмежені на вимірній множині $A$ і ${\forall x\in A\ f(x)\geq g(x)}$, то ${\int\limits_{A} f(x) d x \geq \int\limits_{A} g(x) d x.}$
\end{intextProposition}
\item Модуль інтеграла
\begin{intextProposition}
Якщо функція ${f\colon A \to \R}$ неперервна і обмежена на вимірній множині $A$, то
\[
\left|\int\limits_{A} f(x) d x\right| \leq \int\limits_{A} \left|f(x)\right| d x.
\]
\end{intextProposition}
\end{enumerate}
\begin{remark}
Формулювання теореми про середнє значення у випадку кратних інтегралів по вимірних множинах потребує поняття лінійної зв'язності множин, і тому ми його тут не приводимо.
\end{remark}

Обчислення кратних інтегралів відбувається шляхом зведення них до повторних на підставі наступної теореми.
\begin{theorem}
Нехай задані циліндрична множина $C$, що визначається основою $\ba C$ і функціями ${u_1, u_2\colon \ba C\to \R}$ і функція $f\colon C\to \R$. Якщо
\begin{enumerate}
\item $\ba C$ --- компактна вимірна підмножина \eucl{m-1},
\item функції $u_1$ і $u_2$ неперервні на $\ba C$,
\item функція $f$ неперервна на $C$.
\end{enumerate}
Тоді має місце наступна рівність:
\[
\begin{array}{c}
\int\limits_C f(x)d x = \\
 =\int\limits_{\ba C}\left( \int\limits_{u_1\left(x_1, x_2,\ldots, x_{m-1}\right)}^{u_2\left(x_1, x_2,\ldots, x_{m-1}\right)} f\left(x_1, \ldots, x_{m-1}, x_m \right)d x_m\right) d x_1 \ldots  d x_{m-1}.
\end{array}
\]
\end{theorem}
